\subsection{多级标题}
        如果说页面布局构建了文档的空间,
        那么多级标题(Headings)则构建了文档的逻辑。

        在 \TeX{} 的设计哲学中,
        标题不仅仅是加粗放大的文字,
        它是文档结构的\textbf{锚点}(Anchor)。
        多级标题通过字号(Size)、
        字重(Weight)和垂直间距(Vertical Spacing)的变化,
        建立起清晰的视觉层级(Visual Hierarchy)。
        一个标准的学术文档通常包含以下三级核心结构:
        \begin{enumerate}[label=(\arabic*)]
            \item \textbf{Section(一级标题)}:划分主要的主题模块。
            \item \textbf{Subsection(二级标题)}:阐述主题下的具体分支。
            \item \textbf{Subsubsection(三级标题)}:针对具体技术细节的最小讨论单元。
        \end{enumerate}

        \subsubsection{常用宏包与参数设定}
        原生 \LaTeX{} 的标题样式由 \texttt{article.cls} 定义,
        通常被认为过于呆板且难以修改。
        在现代排版中,\textbf{\texttt{titlesec}} 
        宏包是定制标题样式的良好手段。

        它提供了两个核心命令来控制标题的外观与间距:
        \begin{enumerate}[label=(\arabic*)]
            \item \textbf{\texttt{\textbackslash titleformat}}:控制“长什么样”。
            \begin{lstlisting}[language=TeX, caption={titleformat 命令语法}]
        \titleformat{<命令>}
            [<形状>]          % 如 hang, block, display
            {<格式>}          % 整体字体格式,如 \bfseries
            {<标签>}          % 编号格式,如 \thesection
            {<间隔>}          % 编号与标题文字之间的距离
            {<前代码>}        % 标题文字前的代码
            [<后代码>]        % (可选) 标题后的代码
            \end{lstlisting}
            
            \item \textbf{\texttt{\textbackslash titlespacing}}:控制“位置在哪”。
            \begin{lstlisting}[language=TeX, caption={titlespacing 命令语法}]
        \titlespacing{<命令>}
            {<左间距>}        % 左侧缩进
            {<上间距>}        % 标题上方的垂直空白
            {<下间距>}        % 标题下方的垂直空白
            \end{lstlisting}
        \end{enumerate}

        \subsubsection{本模板的实现}
        在 \texttt{myArticle.cls} 中,
        我们利用 \texttt{titlesec} 宏包对标题进行了调整,
        正文采用了 1.5 倍行距(One-half Spacing),
        标题内部则恢复单倍行距,以避免多行标题显得松散。

        \textbf{源码实现:}
        \begin{lstlisting}[language=TeX]
        % --- 7. 章节标题设置 ---
        \RequirePackage[explicit]{titlesec}

        % 一级标题设置
        \titleformat{\section}
            {\large\bfseries\singlespacing} % 1. 格式:大号、加粗、单倍行距
            {\thesection\quad}              % 2. 标签:编号 + 1个全角空格
            {0pt}                           % 3. 间隔:设为0,由标签自带的空格控制
            {#1}                            % 4. 显式输出标题内容
            []

        \titlespacing{\section}{0pt}{0.5\baselineskip}{0pt}
        \end{lstlisting}
        
        同理,二级(Subsection)和三级标题(Subsubsection)也遵循了相同的逻辑,
        只是字号逐渐减小(\texttt{\textbackslash normalsize}),
        以体现层级递减的视觉比重。

        