\subsection{字体配置}
如果说布局是骨架,那么字体(Typeface)就是文档的皮肤。
在学术排版中,
字体的选择并非为了彰显个性,
而是为了确保跨平台的一致性与长时间阅读的可读性。
        
在西文排版体系中,我们主要关注三大类字体:

        \begin{enumerate}[label=(\arabic*)]
            \item \textbf{衬线体 (Serif/Roman)}:
            笔画起止处有装饰性的“脚”(Serif),笔画粗细有变化(如 Times New Roman, Garamond)。
            \textbf{用途}:正文。衬线有助于引导视线水平移动,是长文阅读的最佳选择。这也是 IEEE, ACM, Springer 等主流学术出版商的强制要求。
            
            \item \textbf{无衬线体 (Sans Serif)}:
            笔画粗细基本均等,无装饰脚(如 Arial, Helvetica)。
            \textbf{用途}:标题、图表标注。因其结构清晰,醒目度高,常用于强调。
            
            \item \textbf{等宽字体 (Monospace/Typewriter)}:
            每个字符占据相同的宽度(如 Courier, Consolas)。
            \textbf{用途}:计算机代码、URL 链接、算法伪代码。
        \end{enumerate}

        \subsubsection{常用实现方案}
        \LaTeX{} 默认使用的字体是 \textbf{Computer Modern (CM)}。虽然经典,但它有两个显著问题:
        \begin{enumerate}[label=(\arabic*)]
            \item \textbf{太细}:在低分辨率屏幕或某些打印机上,CM 字体显得过于纤细,缺乏力量感。
            \item \textbf{风格不符}:大多数工程与自然科学期刊(尤其是航空航天与电子类)更偏爱 \textbf{Times} 风格的字体,因为 Times 比 CM 更节省版面空间,且显得更为紧凑严谨。
        \end{enumerate}

        传统的解决方案是使用 \texttt{mathptmx} 宏包,但它对数学符号的支持非常有限(例如缺少粗体数学符号)。现代排版的最佳实践是使用 \textbf{\texttt{newtx}} 系列宏包,它不仅重构了文本字体,还提供了与 Times 风格完美匹配的数学字库。

        \subsubsection{本模板的实现}
        \texttt{myArticle.cls} 采用了“西文 Times + 中文宋体”的经典组合,这是一种“无需思考但绝对正确”的配置。

        \textbf{源码实现:}
        \begin{lstlisting}[language=TeX]
        % --- 1. 基础宏包 ---
        \RequirePackage[T1]{fontenc}
        \RequirePackage{newtxtext,newtxmath}
        \RequirePackage[scheme=plain, fontset=fandol]{ctex}
        \end{lstlisting}

        \textbf{解析:}
        \begin{enumerate}[label=(\arabic*)]
            \item \textbf{\texttt{newtxtext}}:
            加载基于 TeX Gyre Termes 的正文字体。这是一款高质量的 Times Roman 替代品,它比标准的 Times New Roman 具有更丰富的字重(加粗)和变体。
            
            \item \textbf{\texttt{newtxmath}}:
            这是本模板的精华所在。大多数 Word 用户常犯的错误是:正文用 Times,公式却保留了默认字体(Latin Modern),导致视觉割裂。
            \texttt{newtxmath} 宏包强制将所有的数学符号(希腊字母 $\alpha, \beta$、积分号 $\int$ 等)替换为 Times 风格。请观察公式 $E=mc^2$ 或 $\sum_{i=0}^{n} x_i$,它们的笔触与正文完美融合,浑然一体。
            
            \item \textbf{\texttt{ctex} 与 \texttt{fandol}}:
            \begin{itemize}
                \item \texttt{scheme=plain}:告诉 \texttt{ctex} 宏包“只提供中文支持,不要擅自修改我的字号和标题格式”,从而把控制权留给了我们自己定义的 \texttt{geometry} 和 \texttt{titlesec}。
                \item \texttt{fontset=fandol}:指定使用开源的 Fandol 字体集(FandolSong, FandolHei 等)。这是 \TeX Live 发行版自带的字体,意味着用户\textbf{无需安装任何额外字体}即可在 Windows, Mac 或 Linux 上编译出完全一致的效果。
            \end{itemize}
        \end{enumerate}



