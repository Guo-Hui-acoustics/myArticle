\section{用户接口与命令指南}
    本模板对部分标准命令进行了扩展与预设,
    请按照以下说明使用。

    \subsection{姓名、单位与邮箱信息}
    本模板集成了 \texttt{authblk} 宏包以增强作者信息的排版能力。

    \begin{enumerate}[label=(\arabic*)]
        \item \textbf{基本命令}:\texttt{\textbackslash author}、\texttt{\textbackslash affil} 及 \texttt{\textbackslash email}。
        \item \textbf{排版输出}:使用 \texttt{\textbackslash maketitle} 生成整体标题块。
        \item \textbf{完整示例}:
            \begin{verbatim}
            \title{My Research Title}
            \author{San Zhang}
            % 邮箱接口放到 \affil 内
            \affil{Department of Physics, Xiamen University \email{sanzhang@example.com}}
            \end{verbatim}
    \end{enumerate}

    \subsection{字体样式}
    本模板预置了专业的英文字体(NewTX系列)
    和中文字体支持(Fandol系列),
    标准字体命令的效果如下:

    \begin{table}[h]
        \centering
        \caption{字体命令与效果对照表}
        \begin{tabular}{lll}
            \toprule
            \textbf{命令} & \textbf{说明} & \textbf{模板预设效果} \\
            \midrule
            \texttt{\textbackslash textbf\{...\}} & 加粗 & 英文为 Times 粗体,中文为黑体 \\
            \texttt{\textbackslash textit\{...\}} & 斜体 & 英文为 Times 斜体,中文为楷体 \\
            \texttt{\textbackslash texttt\{...\}} & 等宽 & 打字机字体 (常用于代码片段) \\
            \texttt{\textbackslash textnormal\{...\}} & 正文 & 英文 Times Roman,中文宋体 \\
            \bottomrule
        \end{tabular}
    \end{table}

    \noindent \textbf{注意}:
    模板已加载 \texttt{microtype} 宏包以优化字距,无需手动调整。

    \subsection{代码块环境}
    模板通过 \texttt{listings} 宏包预定义了代码块样式,
    无需手动配置背景色或边框。

    \begin{enumerate}[label=(\arabic*)]
        \item \textbf{环境名称}: \texttt{lstlisting}
        \item \textbf{预设参数}:
        \begin{itemize}
            \item \textbf{背景}:浅灰色 (\texttt{codebg}, gray!0.95)
            \item \textbf{边框}:单实线框 (\texttt{frame=single})
            \item \textbf{字体}:小号等宽字体 (\texttt{\textbackslash small\textbackslash ttfamily})
            \item \textbf{行号}:默认关闭 (\texttt{numbers=none})
            \item \textbf{高亮}:关键字蓝色,字符串红黑色,注释灰色
        \end{itemize}
        \item \textbf{使用示例}:
            \begin{verbatim}
            \begin{lstlisting}[language=Python, caption={Python示例}]
                def main():
                    print("Hello World")
            \end{lstlisting}
            \end{verbatim}
    \end{enumerate}


    \subsection{交叉引用与链接}
    模板整合了 \texttt{natbib} 与 \texttt{hyperref},
    实现了自动化的引用格式与颜色管理。

    \subsubsection{文献引用}
    模板强制使用 AIAA (American Institute of Aeronautics and Astronautics) 样式,需配合提供的 \texttt{new-aiaa.bst} 文件使用。

    \begin{itemize}
        \item \textbf{命令}:\texttt{\textbackslash cite\{key\}},\texttt{key}是参考文献内的关键字 \texttt{label}。
        \item \textbf{效果}:生成绿色方括号数字,如 [1] 或 [1-3]。
    \end{itemize}

    \subsubsection{内部交叉引用}
    支持标准的 \texttt{\textbackslash label} 和 \texttt{\textbackslash ref} 机制。

    \begin{itemize}
        \item \textbf{命令}:\texttt{\textbackslash ref\{sec:intro\}} 或 \texttt{\textbackslash nameref\{sec:intro\}}
        \item \textbf{效果}:生成的引用编号(如 "Section 1")自动附带蓝色超链接,点击可跳转。
    \end{itemize}

    \subsubsection{URL 链接}
    \begin{itemize}
        \item \textbf{命令}:\texttt{\textbackslash url\{...\}} 或 \texttt{\textbackslash href\{url\}\{text\}}
        \item \textbf{效果}:链接文本显示为青色 (Cyan)。
    \end{itemize}