\subsection{页面布局}
        页面布局(Page Layout)是排版设计的物理基础。
        它的本质任务,是在有限的纸张范围内,
        为不同类型的信息分配区域。

        在 \TeX{} 排版系统中,页面并非一张随意的画板,
        而是由无数个矩形“盒子”堆叠而成的严密结构——这被称为
        \textbf{盒子模型}(Box Model)
        \footnote{盒子模型:\TeX{} 系统的核心概念。
        在 \TeX{} 眼中,无论是单个字母、一行文字还是整个段落,
        本质上都是一个具有高度、宽度和深度的矩形盒子。
        排版就是将这些盒子像积木一样拼在一起。}。

        如图 \ref{fig:layout_simple},
            \begin{figure}[htbp]
                \centering
                \includegraphics[width=0.5\textwidth]{images/layout.png} 
                \caption{页面布局的核心区块示意图}
                \label{fig:layout_simple}
            \end{figure}
        一个典型的学术页面由以下四个核心区域组成:
        \begin{enumerate}[label=(\arabic*)]
            \item \textbf{版心 (Body)}:
            这是页面最核心的区域,
            承载了正文流(Main Galley)\footnote{正文流(Main Galley):指尚未分页的、连续的正文内容。就像印刷厂里排在长托盘上的铅字,排版引擎会将其截断并填充到每一页的版心盒子中。}。
            \begin{itemize}
                \item 参数:\texttt{\textbackslash textwidth}(版心宽度)与 \texttt{\textbackslash textheight}(版心高度)。
                这两个参数直接决定了每行能容纳多少字,以及每页能容纳多少行。
            \end{itemize}
            \item \textbf{页眉 (Header)}:
            位于版心上方,通常用于放置章节标题或页码,起到导航作用。
            \begin{itemize}
                \item 参数:\texttt{\textbackslash headheight}(页眉盒子的高度)与 \texttt{\textbackslash headsep}(页眉底部到版心顶部的距离)。
            \end{itemize}
            \item \textbf{页脚 (Footer)}:
            位于版心下方。
            \begin{itemize}
                \item 参数:\texttt{\textbackslash footskip}。
                它定义了从版心底部\textbf{基线}(Baseline)\footnote{基线(Baseline):西文排版中,字母“坐”着的那条看不见的线(例如字母 x 的底部所在的位置,而 p 的尾巴会垂在基线以下)。}到页脚基线的距离。
            \end{itemize}
            \item \textbf{边注区 (Margin Notes)}:
            位于页面的\textbf{切口}(Fore-edge)\footnote{切口(Fore-edge):书页翻开时,与书脊相对的外侧边缘。在单页文档中,通常指左右两侧的空白区域。}处。
            \begin{itemize}
                \item 参数:\texttt{\textbackslash marginparwidth}(边注盒子的宽度)与 \texttt{\textbackslash marginparsep}(边注与版心之间的安全距离)。
            \end{itemize}
        \end{enumerate}

        \subsubsection{常用宏包与参数设定}
        在 \LaTeX{} 的早期版本中,用户需要手动计算上述参数,
        并修改底层的\textbf{长度寄存器}(Length Registers)
        \footnote{长度寄存器:
        \LaTeX{} 内部用于存储物理尺寸(如 10pt, 5cm, 1in)的变量容器。}。这不仅繁琐,而且极易出错(例如,如果左右边距相加超过纸张宽度,内容就会溢出纸面)。

        如今,\textbf{\texttt{geometry}} 宏包已成为业界标准。
        它提供了一套直观的“键值对”(Key-Value)接口,
        自动处理底层复杂的坐标运算。

        最典型的配置方式如下:
        \begin{lstlisting}[language=TeX, caption={geometry 宏包的典型配置}]
        \usepackage{geometry}
        \geometry{
            a4paper,           % 1. 设定物理纸张大小
            left=2.54cm,        % 2. 设定可视边距
            right=2.54cm,
            top=1.91cm,
            bottom=1.91cm
        }
        \end{lstlisting}
        使用该宏包后,用户只需关心“我想留多少空白”,
        而无需关心“版心具体多宽”,
        宏包会自动根据公式 \textbf{版心宽度 = 纸张宽度 - (左边距 + 右边距)} 
        进行反向计算。

        \subsubsection{本模板的实现}
        在 \texttt{myArticle.cls} 中,
        我们采用了较为简单的配置策略,
        力求在标准化的基础上最大化阅读舒适度。

        \textbf{源码实现:}
        \begin{lstlisting}[language=TeX]
        % --- 2. 页面布局 ---
        \RequirePackage[letterpaper,margin=1in]{geometry}
        \end{lstlisting}

        \textbf{设计解析:}
        \begin{enumerate}[label=(\arabic*)]
            \item \textbf{\texttt{letterpaper}}:
            我们将纸张锁定为北美信纸尺寸($8.5 \times 11$ 英寸)。这是绝大多数国际学术期刊(特别是 IEEE、AIAA 等)的默认物理载体。
            
            \item \textbf{\texttt{margin=1in}}:
            这是一条快捷指令,它强制将上、下、左、右四个方向的留白统一设定为 1 英寸。
            
            \item \textbf{版心宽度的推导}:
            在此配置下,版心的实际宽度为:
            $$ 8.5 \text{in} - 1 \text{in} (\text{左}) - 1 \text{in} (\text{右}) = 6.5 \text{in} $$
            配合 10pt 的字号,这能确保每行容纳约 70-80 个英文字符(CPL, Characters Per Line)。在排版心理学中,这是被公认为移动眼球最不易疲劳的“黄金阅读区间”。
        \end{enumerate}
