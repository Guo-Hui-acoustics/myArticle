\section{Quick Start}
\label{sec:setup}
    本章节将详细说明如何使用 \texttt{myArticle.cls} 文档类 / 模板。


    \subsection{获取链接}
    见 \url{https://github.com/Guo-Hui-acoustics/myArticle}


    \subsection{文件组织}
    本模板依赖于特定的文件结构来保证样式与功能的完整性。
    为了确保文档能被正确编译,
    您的工作目录下必须包含以下四个核心文件:
    \begin{enumerate}[label=(\arabic*)]
        \item \textbf{\texttt{myArticle.cls}}: \textbf{文档类文件,即模板。}
        该文件定义了
        页面的几何尺寸(Geometry)、
        基于 \texttt{newtxtext} 的西文字体配置、
        基于 \texttt{ctex} 的中文字体支持,
        以及代码高亮(Listings)等视觉元素。
        \item \textbf{\texttt{new-aiaa.bst}}: \textbf{参考文献样式文件。}
        本模板的学术引用规范遵循 AIAA(美国航空航天学会)标准。
        该 \texttt{.bst} 文件由 \texttt{natbib} 宏包调用,
        负责控制参考文献列表的排序、标点以及缩写规则,
        确保引用的专业性 。
        \item \textbf{\texttt{sample.bib}}: \textbf{参考文献数据库示例。}
        这是一个标准的 BibTeX 数据库文件,
        其中包含了演示所需的文献条目,
        如 Vatistas 等人的流体力学论文 \cite{vatistas1986reverse},
        用于验证引用功能的正确性。
        \item \textbf{\texttt{main.tex}}: \textbf{您的内容写入本文件。}
    \end{enumerate}
    为了保证编写的高效明了,建议您采用如下的文件组织方式:
    \begin{lstlisting}[language=bash, caption={推荐的项目文件组织结构}]
    Project-Root/
    ├── main.tex           % [核心] 主控文件,编译入口
    ├── myArticle.cls      % [核心] 文档类定义
    ├── new-aiaa.bst       % [核心] 参考文献样式
    ├── sample.bib         % [核心] 参考文献数据库
    ├── content/           % [文件夹] 存放各章节源码
    │   ├── abstract.tex   % 摘要内容
    │   └── chapter0.tex   % 第0章内容
    └── images/            % [文件夹] 存放图片资源
        └── fig1.png       % 示例图片
    \end{lstlisting}


    \subsection{最小工作示例}
    为了快速验证环境是否配置正确,
    并展示本模板的基本语法结构,
    我们构建了一个最小工作示例(Minimum Working Example, MWE)。
    该示例涵盖了元数据定义、版权声明注入以及正文引用的完整流程。
    \begin{lstlisting}[language=TeX,caption={\texttt{myArticle} 模板的最小工作示例 (main.tex)}]
    \documentclass{myArticle} 
    % 0. 标题区信息填充
    \title{排版入门:基于\texttt{myArticle.cls}模板}
    %% 作者信息,包含姓名、单位、邮箱,按如下格式填充;
    %% 多个作者直接按顺序向下补充即可
    \author{Guo, H.}
    \affil{College of Physical Science and Technology, Xiamen University \email{gh20222734@163.com}}
    % 1. 正文
    \begin{document}
        % 1.1. 生成标题信息
        \maketitle
        % 1.2. 摘要生成(可选)
        \input{content/abstract.tex}
        % 1.3. 版权框 
        \makecopyright
        % 1.4. 目录——目录页不设置页码
        \pagenumbering{gobble}  
        \tableofcontents
        \newpage  
        % 1.5. 页码设置为阿拉伯数字,从 1 开始
        \pagenumbering{arabic}         
        \setcounter{section}{-1}
        % 1.6. 正文内容插入
        \section{Quick Start}
\label{sec:setup}
    本章节将详细说明如何使用 \texttt{myArticle.cls} 文档类 / 模板。


    \subsection{获取链接}
    见 \url{https://github.com/Guo-Hui-acoustics/myArticle}


    \subsection{文件组织}
    本模板依赖于特定的文件结构来保证样式与功能的完整性。
    为了确保文档能被正确编译,
    您的工作目录下必须包含以下四个核心文件:
    \begin{enumerate}[label=(\arabic*)]
        \item \textbf{\texttt{myArticle.cls}}: \textbf{文档类文件,即模板。}
        该文件定义了
        页面的几何尺寸(Geometry)、
        基于 \texttt{newtxtext} 的西文字体配置、
        基于 \texttt{ctex} 的中文字体支持,
        以及代码高亮(Listings)等视觉元素。
        \item \textbf{\texttt{new-aiaa.bst}}: \textbf{参考文献样式文件。}
        本模板的学术引用规范遵循 AIAA(美国航空航天学会)标准。
        该 \texttt{.bst} 文件由 \texttt{natbib} 宏包调用,
        负责控制参考文献列表的排序、标点以及缩写规则,
        确保引用的专业性 。
        \item \textbf{\texttt{sample.bib}}: \textbf{参考文献数据库示例。}
        这是一个标准的 BibTeX 数据库文件,
        其中包含了演示所需的文献条目,
        如 Vatistas 等人的流体力学论文 \cite{vatistas1986reverse},
        用于验证引用功能的正确性。
        \item \textbf{\texttt{main.tex}}: \textbf{您的内容写入本文件。}
    \end{enumerate}
    为了保证编写的高效明了,建议您采用如下的文件组织方式:
    \begin{lstlisting}[language=bash, caption={推荐的项目文件组织结构}]
    Project-Root/
    ├── main.tex           % [核心] 主控文件,编译入口
    ├── myArticle.cls      % [核心] 文档类定义
    ├── new-aiaa.bst       % [核心] 参考文献样式
    ├── sample.bib         % [核心] 参考文献数据库
    ├── content/           % [文件夹] 存放各章节源码
    │   ├── abstract.tex   % 摘要内容
    │   └── chapter0.tex   % 第0章内容
    └── images/            % [文件夹] 存放图片资源
        └── fig1.png       % 示例图片
    \end{lstlisting}


    \subsection{最小工作示例}
    为了快速验证环境是否配置正确,
    并展示本模板的基本语法结构,
    我们构建了一个最小工作示例(Minimum Working Example, MWE)。
    该示例涵盖了元数据定义、版权声明注入以及正文引用的完整流程。
    \begin{lstlisting}[language=TeX,caption={\texttt{myArticle} 模板的最小工作示例 (main.tex)}]
    \documentclass{myArticle} 
    % 0. 标题区信息填充
    \title{排版入门:基于\texttt{myArticle.cls}模板}
    %% 作者信息,包含姓名、单位、邮箱,按如下格式填充;
    %% 多个作者直接按顺序向下补充即可
    \author{Guo, H.}
    \affil{College of Physical Science and Technology, Xiamen University \email{gh20222734@163.com}}
    % 1. 正文
    \begin{document}
        % 1.1. 生成标题信息
        \maketitle
        % 1.2. 摘要生成(可选)
        \input{content/abstract.tex}
        % 1.3. 版权框 
        \makecopyright
        % 1.4. 目录——目录页不设置页码
        \pagenumbering{gobble}  
        \tableofcontents
        \newpage  
        % 1.5. 页码设置为阿拉伯数字,从 1 开始
        \pagenumbering{arabic}         
        \setcounter{section}{-1}
        % 1.6. 正文内容插入
        \section{Quick Start}
\label{sec:setup}
    本章节将详细说明如何使用 \texttt{myArticle.cls} 文档类 / 模板。


    \subsection{获取链接}
    见 \url{https://github.com/Guo-Hui-acoustics/myArticle}


    \subsection{文件组织}
    本模板依赖于特定的文件结构来保证样式与功能的完整性。
    为了确保文档能被正确编译,
    您的工作目录下必须包含以下四个核心文件:
    \begin{enumerate}[label=(\arabic*)]
        \item \textbf{\texttt{myArticle.cls}}: \textbf{文档类文件,即模板。}
        该文件定义了
        页面的几何尺寸(Geometry)、
        基于 \texttt{newtxtext} 的西文字体配置、
        基于 \texttt{ctex} 的中文字体支持,
        以及代码高亮(Listings)等视觉元素。
        \item \textbf{\texttt{new-aiaa.bst}}: \textbf{参考文献样式文件。}
        本模板的学术引用规范遵循 AIAA(美国航空航天学会)标准。
        该 \texttt{.bst} 文件由 \texttt{natbib} 宏包调用,
        负责控制参考文献列表的排序、标点以及缩写规则,
        确保引用的专业性 。
        \item \textbf{\texttt{sample.bib}}: \textbf{参考文献数据库示例。}
        这是一个标准的 BibTeX 数据库文件,
        其中包含了演示所需的文献条目,
        如 Vatistas 等人的流体力学论文 \cite{vatistas1986reverse},
        用于验证引用功能的正确性。
        \item \textbf{\texttt{main.tex}}: \textbf{您的内容写入本文件。}
    \end{enumerate}
    为了保证编写的高效明了,建议您采用如下的文件组织方式:
    \begin{lstlisting}[language=bash, caption={推荐的项目文件组织结构}]
    Project-Root/
    ├── main.tex           % [核心] 主控文件,编译入口
    ├── myArticle.cls      % [核心] 文档类定义
    ├── new-aiaa.bst       % [核心] 参考文献样式
    ├── sample.bib         % [核心] 参考文献数据库
    ├── content/           % [文件夹] 存放各章节源码
    │   ├── abstract.tex   % 摘要内容
    │   └── chapter0.tex   % 第0章内容
    └── images/            % [文件夹] 存放图片资源
        └── fig1.png       % 示例图片
    \end{lstlisting}


    \subsection{最小工作示例}
    为了快速验证环境是否配置正确,
    并展示本模板的基本语法结构,
    我们构建了一个最小工作示例(Minimum Working Example, MWE)。
    该示例涵盖了元数据定义、版权声明注入以及正文引用的完整流程。
    \begin{lstlisting}[language=TeX,caption={\texttt{myArticle} 模板的最小工作示例 (main.tex)}]
    \documentclass{myArticle} 
    % 0. 标题区信息填充
    \title{排版入门:基于\texttt{myArticle.cls}模板}
    %% 作者信息,包含姓名、单位、邮箱,按如下格式填充;
    %% 多个作者直接按顺序向下补充即可
    \author{Guo, H.}
    \affil{College of Physical Science and Technology, Xiamen University \email{gh20222734@163.com}}
    % 1. 正文
    \begin{document}
        % 1.1. 生成标题信息
        \maketitle
        % 1.2. 摘要生成(可选)
        \input{content/abstract.tex}
        % 1.3. 版权框 
        \makecopyright
        % 1.4. 目录——目录页不设置页码
        \pagenumbering{gobble}  
        \tableofcontents
        \newpage  
        % 1.5. 页码设置为阿拉伯数字,从 1 开始
        \pagenumbering{arabic}         
        \setcounter{section}{-1}
        % 1.6. 正文内容插入
        \section{Quick Start}
\label{sec:setup}
    本章节将详细说明如何使用 \texttt{myArticle.cls} 文档类 / 模板。


    \subsection{获取链接}
    见 \url{https://github.com/Guo-Hui-acoustics/myArticle}


    \subsection{文件组织}
    本模板依赖于特定的文件结构来保证样式与功能的完整性。
    为了确保文档能被正确编译,
    您的工作目录下必须包含以下四个核心文件:
    \begin{enumerate}[label=(\arabic*)]
        \item \textbf{\texttt{myArticle.cls}}: \textbf{文档类文件,即模板。}
        该文件定义了
        页面的几何尺寸(Geometry)、
        基于 \texttt{newtxtext} 的西文字体配置、
        基于 \texttt{ctex} 的中文字体支持,
        以及代码高亮(Listings)等视觉元素。
        \item \textbf{\texttt{new-aiaa.bst}}: \textbf{参考文献样式文件。}
        本模板的学术引用规范遵循 AIAA(美国航空航天学会)标准。
        该 \texttt{.bst} 文件由 \texttt{natbib} 宏包调用,
        负责控制参考文献列表的排序、标点以及缩写规则,
        确保引用的专业性 。
        \item \textbf{\texttt{sample.bib}}: \textbf{参考文献数据库示例。}
        这是一个标准的 BibTeX 数据库文件,
        其中包含了演示所需的文献条目,
        如 Vatistas 等人的流体力学论文 \cite{vatistas1986reverse},
        用于验证引用功能的正确性。
        \item \textbf{\texttt{main.tex}}: \textbf{您的内容写入本文件。}
    \end{enumerate}
    为了保证编写的高效明了,建议您采用如下的文件组织方式:
    \begin{lstlisting}[language=bash, caption={推荐的项目文件组织结构}]
    Project-Root/
    ├── main.tex           % [核心] 主控文件,编译入口
    ├── myArticle.cls      % [核心] 文档类定义
    ├── new-aiaa.bst       % [核心] 参考文献样式
    ├── sample.bib         % [核心] 参考文献数据库
    ├── content/           % [文件夹] 存放各章节源码
    │   ├── abstract.tex   % 摘要内容
    │   └── chapter0.tex   % 第0章内容
    └── images/            % [文件夹] 存放图片资源
        └── fig1.png       % 示例图片
    \end{lstlisting}


    \subsection{最小工作示例}
    为了快速验证环境是否配置正确,
    并展示本模板的基本语法结构,
    我们构建了一个最小工作示例(Minimum Working Example, MWE)。
    该示例涵盖了元数据定义、版权声明注入以及正文引用的完整流程。
    \begin{lstlisting}[language=TeX,caption={\texttt{myArticle} 模板的最小工作示例 (main.tex)}]
    \documentclass{myArticle} 
    % 0. 标题区信息填充
    \title{排版入门:基于\texttt{myArticle.cls}模板}
    %% 作者信息,包含姓名、单位、邮箱,按如下格式填充;
    %% 多个作者直接按顺序向下补充即可
    \author{Guo, H.}
    \affil{College of Physical Science and Technology, Xiamen University \email{gh20222734@163.com}}
    % 1. 正文
    \begin{document}
        % 1.1. 生成标题信息
        \maketitle
        % 1.2. 摘要生成(可选)
        \input{content/abstract.tex}
        % 1.3. 版权框 
        \makecopyright
        % 1.4. 目录——目录页不设置页码
        \pagenumbering{gobble}  
        \tableofcontents
        \newpage  
        % 1.5. 页码设置为阿拉伯数字,从 1 开始
        \pagenumbering{arabic}         
        \setcounter{section}{-1}
        % 1.6. 正文内容插入
        \input{content/chapter0.tex}
        \newpage
        % 1.7. 参考文献生成
        \nocite{*}
        \bibliography{sample}
    \end{document}
    \end{lstlisting}
    

    \subsection{编译方式}
    由于 \texttt{myArticle.cls} 在底层调用了 \texttt{ctex} 宏包,
    并指定了 \texttt{[fontset=fandol]} 选项 ,
    为了获得最佳的中文支持与字体渲染效果,
    \textbf{必须使用 XeLaTeX} 引擎进行编译。

    为确保交叉引用(Cross-reference)和参考文献(Bibliography)都能正确生成,
    建议的完整编译链条如下,
    \begin{enumerate}[label=(\arabic*)]
        \item 使用 \texttt{xelatex} 编译主文档(生成 \texttt{.aux} 辅助文件);
        \item 使用 \texttt{bibtex} 处理参考文献(读取 \texttt{new-aiaa.bst} 生成 \texttt{.bbl} 文件);
        \item 再次使用 \texttt{xelatex} 编译(将处理好的参考文献列表插入文档);
        \item 最后一次使用 \texttt{xelatex} 编译(修正所有的页码跳转与引用编号)。
    \end{enumerate}

    如果您使用 VS Code 配合 LaTeX Workshop 插件,
    建议将默认的构建配方(Recipe)设置为 
    \texttt{xelatex -> bibtex -> xelatex*2}。




        \newpage
        % 1.7. 参考文献生成
        \nocite{*}
        \bibliography{sample}
    \end{document}
    \end{lstlisting}
    

    \subsection{编译方式}
    由于 \texttt{myArticle.cls} 在底层调用了 \texttt{ctex} 宏包,
    并指定了 \texttt{[fontset=fandol]} 选项 ,
    为了获得最佳的中文支持与字体渲染效果,
    \textbf{必须使用 XeLaTeX} 引擎进行编译。

    为确保交叉引用(Cross-reference)和参考文献(Bibliography)都能正确生成,
    建议的完整编译链条如下,
    \begin{enumerate}[label=(\arabic*)]
        \item 使用 \texttt{xelatex} 编译主文档(生成 \texttt{.aux} 辅助文件);
        \item 使用 \texttt{bibtex} 处理参考文献(读取 \texttt{new-aiaa.bst} 生成 \texttt{.bbl} 文件);
        \item 再次使用 \texttt{xelatex} 编译(将处理好的参考文献列表插入文档);
        \item 最后一次使用 \texttt{xelatex} 编译(修正所有的页码跳转与引用编号)。
    \end{enumerate}

    如果您使用 VS Code 配合 LaTeX Workshop 插件,
    建议将默认的构建配方(Recipe)设置为 
    \texttt{xelatex -> bibtex -> xelatex*2}。




        \newpage
        % 1.7. 参考文献生成
        \nocite{*}
        \bibliography{sample}
    \end{document}
    \end{lstlisting}
    

    \subsection{编译方式}
    由于 \texttt{myArticle.cls} 在底层调用了 \texttt{ctex} 宏包,
    并指定了 \texttt{[fontset=fandol]} 选项 ,
    为了获得最佳的中文支持与字体渲染效果,
    \textbf{必须使用 XeLaTeX} 引擎进行编译。

    为确保交叉引用(Cross-reference)和参考文献(Bibliography)都能正确生成,
    建议的完整编译链条如下,
    \begin{enumerate}[label=(\arabic*)]
        \item 使用 \texttt{xelatex} 编译主文档(生成 \texttt{.aux} 辅助文件);
        \item 使用 \texttt{bibtex} 处理参考文献(读取 \texttt{new-aiaa.bst} 生成 \texttt{.bbl} 文件);
        \item 再次使用 \texttt{xelatex} 编译(将处理好的参考文献列表插入文档);
        \item 最后一次使用 \texttt{xelatex} 编译(修正所有的页码跳转与引用编号)。
    \end{enumerate}

    如果您使用 VS Code 配合 LaTeX Workshop 插件,
    建议将默认的构建配方(Recipe)设置为 
    \texttt{xelatex -> bibtex -> xelatex*2}。




        \newpage
        % 1.7. 参考文献生成
        \nocite{*}
        \bibliography{sample}
    \end{document}
    \end{lstlisting}
    

    \subsection{编译方式}
    由于 \texttt{myArticle.cls} 在底层调用了 \texttt{ctex} 宏包,
    并指定了 \texttt{[fontset=fandol]} 选项 ,
    为了获得最佳的中文支持与字体渲染效果,
    \textbf{必须使用 XeLaTeX} 引擎进行编译。

    为确保交叉引用(Cross-reference)和参考文献(Bibliography)都能正确生成,
    建议的完整编译链条如下,
    \begin{enumerate}[label=(\arabic*)]
        \item 使用 \texttt{xelatex} 编译主文档(生成 \texttt{.aux} 辅助文件);
        \item 使用 \texttt{bibtex} 处理参考文献(读取 \texttt{new-aiaa.bst} 生成 \texttt{.bbl} 文件);
        \item 再次使用 \texttt{xelatex} 编译(将处理好的参考文献列表插入文档);
        \item 最后一次使用 \texttt{xelatex} 编译(修正所有的页码跳转与引用编号)。
    \end{enumerate}

    如果您使用 VS Code 配合 LaTeX Workshop 插件,
    建议将默认的构建配方(Recipe)设置为 
    \texttt{xelatex -> bibtex -> xelatex*2}。



