
\subsection{行距与段落间距}
        文字的垂直排列密度直接影响阅读的疲劳度。
        过于紧密的行距会使视线迷失(难以换行),
        而过于稀疏的行距则会割裂阅读的连续性。

        在排版术语中,我们需要区分两个维度的垂直间距:
        \begin{enumerate}[label=(\arabic*)]
            \item \textbf{行距 (Line Spacing/Leading)}:
            指相邻两行文字\textbf{基线}(Baseline)之间的垂直距离。
            在 \TeX{} 底层,它由 \texttt{\textbackslash baselineskip} 控制。一般而言,正文行距设为字号的 1.2 至 1.5 倍最为适宜。例如,对于 10pt 的文字,12pt 的行距是标准设定(单倍行距),而 14.5pt 至 15pt 则能带来更宽松的阅读体验。
            \item \textbf{段落间距 (Paragraph Spacing)}:
            指前后两个段落之间的额外垂直空白。
            在 \TeX{} 中,由 \texttt{\textbackslash parskip} 控制。学术排版通常存在两种流派:
            \begin{itemize}
                \item \textbf{缩进式 (Indented)}:段落首行缩进,段间无额外空行(传统书籍风格)。
                \item \textbf{间距式 (Spaced)}:段落首行无缩进,段间增加空行(现代网页或商务文档风格)。
            \end{itemize}
        \end{enumerate}

        \subsubsection{常用宏包与参数设定}
        虽然我们可以通过暴力修改 \texttt{\textbackslash baselineskip} 来改变行距,
        但这是一种“危险”的做法,
        因为它会无差别地影响页眉、页脚、脚注甚至公式内部的间距。

        \textbf{1. 行距的现代解决方案:\texttt{setspace} 宏包}

        \texttt{setspace} 是管理行距的行业标准。
        它能智能地只调整正文行距,而保持表格、脚注等元素的单倍行距,
        避免排版混乱。

        常用命令包括:
        \begin{lstlisting}[language=TeX, caption={setspace 宏包的常用命令}]
        \usepackage{setspace}
        \singlespacing      % 单倍行距 (默认)
        \onehalfspacing     % 1.5 倍行距
        \doublespacing      % 2.0 倍行距
        \setstretch{1.25}   % 自定义倍数
        \end{lstlisting}

        \textbf{2. 段落间距的解决方案}

        如果需要“间距式”排版,
        不建议直接修改 \texttt{\textbackslash parskip},
        推荐使用 \textbf{\texttt{parskip}} 宏包。
        它会自动处理列表(List)和标题周围的垂直间距,
        防止页面显得支离破碎。

        \subsubsection{本模板的实现}
        \texttt{myArticle.cls} 倾向于一种“易于批改和阅读”的草稿风格,
        因此我们在保持传统缩进的同时,显著增加了行距。

        \textbf{源码实现:}
        \begin{lstlisting}[language=TeX]
        % --- 2. 页面布局 (部分) ---
        \RequirePackage{setspace}
        \onehalfspacing 
        \end{lstlisting}

        \textbf{设计解析:}
        \begin{enumerate}
            \item \textbf{\texttt{\textbackslash onehalfspacing}}:
            我们强制开启了 1.5 倍行距。
            这在学术投稿(特别是初稿)中非常常见,
            因为宽大的行间距为审稿人留出了手写批注的空间,
            同时也降低了屏幕阅读时长文带来的视觉压力。
            
            \item \textbf{关于段落间距的取舍}:
            本模板\textbf{未}引入 \texttt{parskip} 宏包,
            也未修改 \texttt{\textbackslash parskip} 的默认值。
            这意味着我们遵循 \LaTeX{} 标准文档类(article)的默认行为:\textbf{使用首行缩进来区分段落,段落之间没有额外的垂直空白}。
            这种紧凑的段落处理方式与 1.5 倍的行距形成互补——行内宽松,段间紧凑,
            维持了版面的灰度平衡。
        \end{enumerate}