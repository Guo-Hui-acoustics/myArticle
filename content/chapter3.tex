\section{参考文献管理}
\label{sec:bibliography}

在学术写作中,参考文献的规范性与正文同等重要。
本模板采用经典的 \textbf{BibTeX} 引擎进行文献管理,
实现了“内容与格式分离”的高效工作流。

\subsection{基本格式与原理}
传统的排版软件(如 Word)通常要求作者手动调整每一条参考文献的格式(如斜体、句号位置等)。
而在 \LaTeX{} 中,
我们只需维护一个结构化的数据库文件(\texttt{.bib}),
格式渲染完全交给样式文件(\texttt{.bst})处理。

用户需要在一个独立的 \texttt{.bib} 文件中存储文献信息。
每条文献由一个唯一的引用键(Key)标识。

\textbf{示例(sample.bib):}
\begin{lstlisting}[language=TeX]
@article{vatistas1986reverse,
  title={Reverse Flow Radius in Vortex Chambers},
  author={Vatistas, G H and Lin, S and Kwok, C K},
  journal={AIAA Journal},
  volume={24},
  number={11},
  pages={1872, 1873},
  year={1986},
  publisher={AIAA}
}
\end{lstlisting}
在这个例子中,\textbf{key} 就是 \texttt{vatistas1986reverse},
作为我们在正文中引用参考文献的\textbf{标签}。

\subsection{本模板的实现与接口}
本模板旨在符合工程领域的严谨标准,因此在底层强绑定了 AIAA 的引用样式。

\subsubsection{实现机制}
在 \texttt{myArticle.cls} 中,我们通过以下核心代码配置了引用环境:

\begin{lstlisting}[language=TeX]
% --- 9. 引用与链接 ---
\RequirePackage[sort&compress,numbers]{natbib}
\bibliographystyle{new-aiaa}
\renewcommand{\bibfont}{\small}
\end{lstlisting}

\textbf{代码解析:}
\begin{enumerate}[label=(\arabic*)]
    \item \textbf{\texttt{natbib} 宏包}:
    加载了强大的 \texttt{natbib} 宏包,并开启了两个关键选项:
    \begin{itemize}
        \item \texttt{numbers}:强制使用数字引用风格(如 [1]),而非“作者-年份”风格。
        \item \texttt{sort\&compress}:智能排序与压缩。例如,将分散的引用 [1, 3, 2] 自动整理为 [1-3]。
    \end{itemize}
    
    \item \textbf{\texttt{new-aiaa} 样式}:
    指定使用模板自带的 \texttt{new-aiaa.bst} 文件。该文件严格定义了 AIAA 期刊的排版规则(如期刊名缩写、年份位置等)。
    
    \item \textbf{字体调整}:
    将参考文献列表的字号缩小为 \texttt{\textbackslash small},这是学术界通用的做法,以区分正文与附录。
\end{enumerate}

\subsubsection{使用接口}
用户在使用时无需关心复杂的格式定义,只需通过简单的命令即可完成调用。

\textbf{1. 正文引用接口:}
\begin{itemize}
    \item \textbf{命令}:\texttt{\textbackslash cite\{Key\}}
    \item \textbf{示例}:根据 Vatistas 的研究 \cite{vatistas1986reverse}...
    \item \textbf{效果}:根据 Vatistas 的研究 [1]... (数字会自动变为\textcolor{green}{绿色}链接)
\end{itemize}

\textbf{2. 列表生成接口:}
在文档末尾(\texttt{\textbackslash end\{document\}} 之前),使用以下命令生成列表:

\begin{lstlisting}[language=TeX]
\bibliography{sample}  % 指定 .bib 文件名(不带扩展名)
\end{lstlisting}


\textbf{注意}:未在正文中被引用的条目默认不会出现在列表中。如果需要列出所有文献(无论是否引用),可使用 \texttt{\textbackslash nocite\{*\}} 命令。